\PassOptionsToPackage{unicode=true}{hyperref} % options for packages loaded elsewhere
\PassOptionsToPackage{hyphens}{url}
%
\documentclass[ignorenonframetext,]{beamer}
\usepackage{pgfpages}
\setbeamertemplate{caption}[numbered]
\setbeamertemplate{caption label separator}{: }
\setbeamercolor{caption name}{fg=normal text.fg}
\beamertemplatenavigationsymbolsempty
% Prevent slide breaks in the middle of a paragraph:
\widowpenalties 1 10000
\raggedbottom
\setbeamertemplate{part page}{
\centering
\begin{beamercolorbox}[sep=16pt,center]{part title}
  \usebeamerfont{part title}\insertpart\par
\end{beamercolorbox}
}
\setbeamertemplate{subsection page}{
\centering
\begin{beamercolorbox}[sep=8pt,center]{part title}
  \usebeamerfont{subsection title}\insertsubsection\par
\end{beamercolorbox}
}
\AtBeginPart{
  \frame{\partpage}
}
\AtBeginSection{
  \ifbibliography
  \else
    %\frame{\sectionpage}
  \begin{frame}
  \vfill
  \centering
  \begin{beamercolorbox}[sep=8pt,center,shadow=true,rounded=true]{title}
    \usebeamerfont{title}\insertsectionhead\par%
  \end{beamercolorbox}
  \vfill
  \end{frame}
  \fi
}
\AtBeginSubsection{
  \frame{\subsectionpage}
}
\usepackage{lmodern}
\usepackage{amssymb,amsmath}
\usepackage{ifxetex,ifluatex}
\usepackage{fixltx2e} % provides \textsubscript
\ifnum 0\ifxetex 1\fi\ifluatex 1\fi=0 % if pdftex
  \usepackage[T1]{fontenc}
  \usepackage[utf8]{inputenc}
  \usepackage{textcomp} % provides euro and other symbols
\else % if luatex or xelatex
  \usepackage{unicode-math}
  \defaultfontfeatures{Ligatures=TeX,Scale=MatchLowercase}
\fi
\usetheme[]{Warsaw}
\usecolortheme{beaver}
\useoutertheme{infolines}
% use upquote if available, for straight quotes in verbatim environments
\IfFileExists{upquote.sty}{\usepackage{upquote}}{}
% use microtype if available
\IfFileExists{microtype.sty}{%
\usepackage[]{microtype}
\UseMicrotypeSet[protrusion]{basicmath} % disable protrusion for tt fonts
}{}
\IfFileExists{parskip.sty}{%
\usepackage{parskip}
}{% else
\setlength{\parindent}{0pt}
\setlength{\parskip}{6pt plus 2pt minus 1pt}
}
\usepackage{hyperref}
\hypersetup{
            pdftitle={Presentation template (long title)},
            pdfauthor={Author Name},
            pdfborder={0 0 0},
            breaklinks=true}
\urlstyle{same}  % don't use monospace font for urls
\newif\ifbibliography
\usepackage{color}
\usepackage{fancyvrb}
\newcommand{\VerbBar}{|}
\newcommand{\VERB}{\Verb[commandchars=\\\{\}]}
\DefineVerbatimEnvironment{Highlighting}{Verbatim}{commandchars=\\\{\}}
% Add ',fontsize=\small' for more characters per line
\newenvironment{Shaded}{}{}
\newcommand{\AlertTok}[1]{\textcolor[rgb]{1.00,0.00,0.00}{\textbf{#1}}}
\newcommand{\AnnotationTok}[1]{\textcolor[rgb]{0.38,0.63,0.69}{\textbf{\textit{#1}}}}
\newcommand{\AttributeTok}[1]{\textcolor[rgb]{0.49,0.56,0.16}{#1}}
\newcommand{\BaseNTok}[1]{\textcolor[rgb]{0.25,0.63,0.44}{#1}}
\newcommand{\BuiltInTok}[1]{#1}
\newcommand{\CharTok}[1]{\textcolor[rgb]{0.25,0.44,0.63}{#1}}
\newcommand{\CommentTok}[1]{\textcolor[rgb]{0.38,0.63,0.69}{\textit{#1}}}
\newcommand{\CommentVarTok}[1]{\textcolor[rgb]{0.38,0.63,0.69}{\textbf{\textit{#1}}}}
\newcommand{\ConstantTok}[1]{\textcolor[rgb]{0.53,0.00,0.00}{#1}}
\newcommand{\ControlFlowTok}[1]{\textcolor[rgb]{0.00,0.44,0.13}{\textbf{#1}}}
\newcommand{\DataTypeTok}[1]{\textcolor[rgb]{0.56,0.13,0.00}{#1}}
\newcommand{\DecValTok}[1]{\textcolor[rgb]{0.25,0.63,0.44}{#1}}
\newcommand{\DocumentationTok}[1]{\textcolor[rgb]{0.73,0.13,0.13}{\textit{#1}}}
\newcommand{\ErrorTok}[1]{\textcolor[rgb]{1.00,0.00,0.00}{\textbf{#1}}}
\newcommand{\ExtensionTok}[1]{#1}
\newcommand{\FloatTok}[1]{\textcolor[rgb]{0.25,0.63,0.44}{#1}}
\newcommand{\FunctionTok}[1]{\textcolor[rgb]{0.02,0.16,0.49}{#1}}
\newcommand{\ImportTok}[1]{#1}
\newcommand{\InformationTok}[1]{\textcolor[rgb]{0.38,0.63,0.69}{\textbf{\textit{#1}}}}
\newcommand{\KeywordTok}[1]{\textcolor[rgb]{0.00,0.44,0.13}{\textbf{#1}}}
\newcommand{\NormalTok}[1]{#1}
\newcommand{\OperatorTok}[1]{\textcolor[rgb]{0.40,0.40,0.40}{#1}}
\newcommand{\OtherTok}[1]{\textcolor[rgb]{0.00,0.44,0.13}{#1}}
\newcommand{\PreprocessorTok}[1]{\textcolor[rgb]{0.74,0.48,0.00}{#1}}
\newcommand{\RegionMarkerTok}[1]{#1}
\newcommand{\SpecialCharTok}[1]{\textcolor[rgb]{0.25,0.44,0.63}{#1}}
\newcommand{\SpecialStringTok}[1]{\textcolor[rgb]{0.73,0.40,0.53}{#1}}
\newcommand{\StringTok}[1]{\textcolor[rgb]{0.25,0.44,0.63}{#1}}
\newcommand{\VariableTok}[1]{\textcolor[rgb]{0.10,0.09,0.49}{#1}}
\newcommand{\VerbatimStringTok}[1]{\textcolor[rgb]{0.25,0.44,0.63}{#1}}
\newcommand{\WarningTok}[1]{\textcolor[rgb]{0.38,0.63,0.69}{\textbf{\textit{#1}}}}
\setlength{\emergencystretch}{3em}  % prevent overfull lines
\providecommand{\tightlist}{%
  \setlength{\itemsep}{0pt}\setlength{\parskip}{0pt}}
\setcounter{secnumdepth}{0}

% set default figure placement to htbp
\makeatletter
\def\fps@figure{htbp}
\makeatother

\newlength{\cslhangindent}
\setlength{\cslhangindent}{1.5em}
\newenvironment{cslreferences}%
  {}%
  {\par}

\title[template (short title)]{Presentation template (long title)}
\author{Author Name}
\providecommand{\institute}[1]{}
\institute[INST1, INST2]{Institute 1 \and Institute 2 \newline on two lines}
\date{\today}

\begin{document}
\frame{\titlepage}

\hypertarget{section-title}{%
\section{Section title}\label{section-title}}

\begin{frame}{Slide title}
\protect\hypertarget{slide-title}{}
Some content:

\begin{itemize}
\tightlist
\item
  See {[}\protect\hyperlink{ref-gptNIPS}{1}{]} (clickable reference)

  \begin{enumerate}
  \tightlist
  \item
    This means you screwed up the references: {[}{\textbf{???}}{]}
  \end{enumerate}
\item
  \emph{Italic}

  \begin{itemize}
  \tightlist
  \item
    \textbf{Bold}
  \end{itemize}
\end{itemize}

Vertical space to better separate things aa
\end{frame}

\hypertarget{section-2}{%
\section{Section 2}\label{section-2}}

\begin{frame}{A fictional theoretical basis}
\protect\hypertarget{a-fictional-theoretical-basis}{}
An equation to appear smart:

\begin{equation}
\mathcal{L}=-\frac 1 N \sum_{i=1}^N \log p(y_i|x_i,\theta)+\sum_{i=1}^L \frac{\lambda^2 p^{keep}_i}{2N}||\theta_i||_2^2
\label{eq1}
\end{equation}
\end{frame}

\begin{frame}[fragile]{Pretending that something was actually done}
\protect\hypertarget{pretending-that-something-was-actually-done}{}
BAYESIAN DEEP LEARNING IN ONE SLIDE !!1!1!!1

\begin{Shaded}
\begin{Highlighting}[]
\NormalTok{l2 }\OperatorTok{=}\NormalTok{ (}\DecValTok{1} \OperatorTok{{-}}\NormalTok{ pdrop) }\OperatorTok{*}\NormalTok{ length\_scale\_squared }\OperatorTok{/}\NormalTok{ (}\DecValTok{2} \OperatorTok{*}\NormalTok{ bsize)}

\NormalTok{inp }\OperatorTok{=}\NormalTok{ x }\OperatorTok{=}\NormalTok{ layers.Input(shape}\OperatorTok{=}\NormalTok{(}\DecValTok{2}\NormalTok{,))  }\CommentTok{\# input}
\NormalTok{x }\OperatorTok{=}\NormalTok{ layers.Dense(}\DecValTok{200}\NormalTok{, activation}\OperatorTok{=}\StringTok{\textquotesingle{}relu\textquotesingle{}}\NormalTok{,}
\NormalTok{                 kernel\_regularizer}\OperatorTok{=}\NormalTok{regularizers.l2(l2))(x)}
\NormalTok{x }\OperatorTok{=}\NormalTok{ layers.Dropout(pdrop)(x, training}\OperatorTok{=}\VariableTok{True}\NormalTok{)}
\NormalTok{x }\OperatorTok{=}\NormalTok{ layers.Dense(}\DecValTok{200}\NormalTok{, activation}\OperatorTok{=}\StringTok{\textquotesingle{}sigmoid\textquotesingle{}}\NormalTok{,}
\NormalTok{                 kernel\_regularizer}\OperatorTok{=}\NormalTok{regularizers.l2(l2))(x)}
\NormalTok{x }\OperatorTok{=}\NormalTok{ layers.Dropout(pdrop)(x, training}\OperatorTok{=}\VariableTok{True}\NormalTok{)}
\NormalTok{out }\OperatorTok{=}\NormalTok{ layers.Dense(}\DecValTok{2}\NormalTok{, activation}\OperatorTok{=}\StringTok{\textquotesingle{}tanh\textquotesingle{}}\NormalTok{,}
\NormalTok{                   kernel\_regularizer}\OperatorTok{=}\NormalTok{regularizers.l2(l2))(x)}
\end{Highlighting}
\end{Shaded}
\end{frame}

\hypertarget{section-3}{%
\section{Section 3}\label{section-3}}

\begin{frame}{A nice centered picture}
\protect\hypertarget{a-nice-centered-picture}{}
\centering\includegraphics[width=0.5\linewidth,height=0.5\textheight]{example-image-a}
\end{frame}

\begin{frame}{``Layouts''}
\protect\hypertarget{layouts}{}
Two-columns content following this very long line that is in a single column and wraps:

\begin{columns}
\column{0.6\textwidth}

Fancy linear models (unlike Eq. \ref{eq1})

$$
\log\frac{p(C_p|a_{p-4},\ldots,a_{p+1})}{p(C_p)}
= \sum_{i=-4}^1 \phi(a_{p+i}, i)
$$

\column{0.4\textwidth}

\begin{figure}
\centering\includegraphics[width=0.95\linewidth]{example-image-b}
\caption{A figure}
\label{fig}
\end{figure}

\begin{table}
\begin{tabular}{cc}
a & b \\
\hline
1 & 2 \\
3 & 4 \\
\end{tabular}
\caption{A table (also see Fig. \ref{fig})}
\end{table}

\end{columns}
\end{frame}

\hypertarget{thank-you}{%
\section{Thank you!}\label{thank-you}}

\hypertarget{references}{%
\section*{References}\label{references}}
\addcontentsline{toc}{section}{References}

\begin{frame}[allowframebreaks]{References}
\hypertarget{refs}{}
\begin{cslreferences}
\leavevmode\hypertarget{ref-gptNIPS}{}%
1. GPT-5. A very important discovery about deep learning. In: Advances in neural information processing systems 42+\(\pi\). pp. --3--14.
\end{cslreferences}
\end{frame}

\end{document}
